\documentclass[%
a4paper, %apaper,   % alle weiteren Papierformat einstellbar
%empty,             % keine Seitenzahlen
9pt,               % Schriftgröße (12pt, 11pt (Standard))
leqno,              % Nummerierung von Gleichungen links
fleqn,              % Ausgabe von Gleichungen linksbündig
%smallheadings,
%abstracton,
]
{scrartcl}

%% Deutsche Anpassungen
\usepackage[ngerman]{babel}

\usepackage{xltxtra, fontspec, xunicode}

\newcommand{\headerfont}{Gill Sans Std}

% TODO: Die Schrift auf eine ändern, die man auch auf dem System findet.
%\setmainfont[Mapping=tex-text]{Palatino LT Std}
\setmainfont[Mapping=tex-text]{Minion Pro}
\setkomafont{sectioning}{\fontspec{\headerfont}}
\setkomafont{title}{\fontspec{\headerfont}}
%\linespread{1.05}

%obligatorischer Mathekram:
\usepackage{amssymb,amstext}
%\usepackage{stmaryrd}
\usepackage[sumlimits]{amsmath}
\usepackage{eulervm}
%\usepackage{mnsymbol}

%nützliche Extras:
\usepackage{array,
  %hhline,
  %tabularx,
  %enumerate,
  %color,
  %setspace,
  %booktabs,
  %cite,
  %caption,
  %lineno,
  %lastpage,
  %algorithm,
  %algorithmic,
  listings,
  %bussproofs,
  %ulem,
  %multirow,
  %natbib,
  %url,
}
%\usepackage{ltxtable}
\usepackage[xetex, colorlinks=false, pdfnewwindow=true]{hyperref}

\usepackage[
  cm,
  headings
]{fullpage}
%\usepackage[left=1cm,right=1cm,top=1cm,bottom=1cm,includeheadfoot]{geometry}

\lstset{
  language=bash,
  basicstyle=\fontspec{\codefont},
  showstringspaces=false,
  keywordstyle=\color{blue}\bfseries,
}

\usepackage{fancyhdr}
\pagestyle{fancy}
\fancyhf{}

% Veranstaltung:
\fancyhead[L]{\fontspec{\headerfont}Paradigmen der Programmierung}
\fancyhead[C]{}
\fancyhead[R]{\fontspec{\headerfont}4. April 2012}

% \fancyfoot[L]{}
\fancyfoot[C]{\thepage}
% \fancyfoot[R]{\thepage / \pageref{LastPage}}

%% Linie oben/unten
\renewcommand{\headrulewidth}{0.0pt}
%\renewcommand{\footnoterule}{}

\parindent 0pt

%% Pakete für Grafiken & Abbildungen
%\usepackage{graphicx}
%\usepackage{tikz}      %%TeX ist kein Zeichenprogramm
%\usetikzlibrary{calc,arrows}

%% Lochmarken; auskommentieren, falls unerwünscht
%\usepackage{eso-pic}
%\AddToShipoutPicture*{
%    \put(0,68.5mm){\rule{15pt}{0.5pt}}
%    \put(0,0.5\paperheight){\rule{20pt}{0.5pt}}
%    \put(0,228.5mm){\rule{15pt}{0.5pt}}
%}

%\usepackage{pdflscape}

\newcommand{\codefont}{Inconsolata}
%\newcommand{\codefont}{Bitstream Vera Sans Mono}
\newcommand{\code}[1]{{\fontspec[Mapping=tex-text]{\codefont}#1}}
\newcommand{\tinycode}[1]{{\fontspec[Mapping=tex-text]{\codefont}\tiny#1}}

\begin{document}

\section*{Organisatorisches}
\begin{itemize}
\item Übungsleiter: Albert Hein
\item Modulprüfung: Klausur:
  \begin{itemize}
    \item 31. Juli 2012, 9-11 Uhr
    \item KZ-Haus (A.-E.-Str. 22), HS037
    \item Zulassungsbedingung: 75\% der Übungsaufgaben
  \end{itemize}
\item Note = 100\% der Klausur
\item bezüglich der Übungen:
  \begin{itemize}
    \item Ort: Raum 219, Sun-Pool
    \item Start: ab dem 16. April
    \item wichtig: Accounts klar Schiff machen!
    \item Termine:
      \begin{itemize}
        \item immer in den geraden Wochen
        \item erster Termin: 17.4. bzw. 20.4.
        \item dienstags 15:15 - 16:45 Uhr
        \item freitags 9:00 - 10:30 Uhr
      \end{itemize}
  \end{itemize}
\end{itemize}

\section*{Shells \& Tools}
\begin{itemize}
\item zum Warmwerden ein Beispiel mit vielen Zahlen
\item Programm \code{summary}, das spaltenweise Mittelwerte und
  Standardabweichung berechnet
\item ergibt $793 \cdot 2 = 1586$ Zahlen
\item Quizfrage: wie wird die Eingabedatei \code{numbers.dat} mit \code{summary}
  verarbeitet?
  \begin{itemize}
    \item Lösung 1: massive manpower
    \item Lösung 2: mit dem Computer: \code{\$ summary < numbers.dat}
  \end{itemize}
\item Was tun bei Kommentarzeilen?
  \begin{itemize}
    \item erste Kommentarzeile rausfiltern (z.\,B. mit \code{head}: \code{\$
      head -1 < numbers.dat | summary})
    \item Datenfluss: \code{numbers.dat} $\rightarrow$ \code{head} $\rightarrow$
      \code{summary}
    \item im schlimmsten Fall werden alle Daten erst komplett durch \code{head}
      gejagt und dann nochmal durch \code{summary}
  \end{itemize}
\item Wir wollen aber, dass jede Zeile nur ein einziges Mal gelesen wird! Was
  tun?
  \begin{lstlisting}
#!/bin/bash
read
exec summary
  \end{lstlisting}
  siehe auch: \url{http://man.cx/read}, \url{http://man.cx/exec}
\item Aufruf: \code{\$ butfirst.sh < numbers.dat}
\end{itemize}

\section*{Fallstudie I: DZNE - Sensor based monitoring}
\begin{itemize}
\item Worum geht's?
  \begin{itemize}
    \item „neurodegenerative Erkrankungen“ werden gerade in der westlichen Welt
      zum großen Problem
    \item Kriegen wir eine Sensortechnik hin, die Früherkennung leistet?
  \end{itemize}
\item bei der Verarbeitung veranstaltet man ganz viel Datenfoo
\item Bezug zur Vorlesung: Datenstromverarbeitung mit AWK
\item Problem: Daten werden meist in voller Gänze in den Speicher geladen und
  Dinge\texttrademark\ werden sehr schnell sehr langsam
\item für die Prozesssteuerung wurden Shell Skripte eingesetzt
\end{itemize}

\section*{Prozesspipelines \& Glue Languages}
\begin{itemize}
\item Unix - Wiederholung/Designphilosophie
  \begin{itemize}
  \item Demotime:
    \begin{lstlisting}
$ ls -l
$ echo 'Hello Welt'
    \end{lstlisting}
    siehe auch: \href{http://de.wikipedia.org/wiki/Ls_(Unix)}{WP: ls (Unix)},
    \href{http://de.wikipedia.org/wiki/Echo_(Informatik)}{WP: echo (Informatik)}
  \item Kurzeinführung: was ist ein Betriebssystem? (siehe auch:
    \href{https://de.wikipedia.org/wiki/Betriebssystem}{WP: Betriebssysteme})
  \item Was ist ein Prozess?
    \begin{itemize}
      \item Programm in der Ausführung
      \item aus Sicht des OS: Eintrag in der Prozesstabelle
      \item siehe auch:
        \href{http://de.wikipedia.org/wiki/Prozess_(Informatik)}{WP: Prozess
        (Informatik)}
    \end{itemize}
  \item monolithisches OS vs. Microkernel
    \begin{itemize}
      \item siehe auch:
        \href{https://de.wikipedia.org/wiki/Monolithischer_Kernel}{WP:
        Monolithischer Kernel},
        \href{https://de.wikipedia.org/wiki/Microkernel}{WP: Microkernel}
    \end{itemize}
  \item Demotime:
    \begin{lstlisting}
$ ls -l
$ tr [:lower:] [:upper:]
$ ls -l | tr [:lower:] [:upper:]
    \end{lstlisting}
    siehe auch: \href{http://en.wikipedia.org/wiki/Tr_(Unix)}{eWP: tr (Unix)}
  \item (fast) alles ist eine Datei
    \begin{itemize}
      \item Zugriff mittels File-Deskriptor
      \item siehe auch: \href{https://de.wikipedia.org/wiki/Datei-Handle}{WP:
        Datei-Handle}, 
        \href{https://de.wikipedia.org/wiki/Standard-Datenströme}{WP:
        Standard-Datenströme}
    \end{itemize}
  \item Demotime:
    \begin{lstlisting}
$ tr [:lower:] [:upper:] < testargs.sh
    \end{lstlisting}
  \end{itemize}
\item Hierarchisches Dateisystem: keine weitere Erwähnung
\item Kernel und Prozesse (Fork, Exec, File-Deskriptoren)
  \begin{itemize}
    \item Ist die Shell was besonderes?
    \item[$\Rightarrow$] Nein! Sie ist ein ganz normaler Prozess.
    \item wird später genauer drauf eingegangen
  \end{itemize}
\end{itemize}

\section*{Beispiel 2: „Tischrechner“}
\begin{itemize}
\item Was tut das Beispielskript?
  \begin{itemize}
    \item in der while-Schleife wird ein Ausdruck eingegeben
    \item Hilfsfunktion \code{calculate} löscht ein paar Dateien
    \item schreibt in die Datei \code{mycalc.c} ein C-Programm, das den Ausdruck
      berechnet
    \item dieses C-Programm wird kompiliert und ausgeführt
  \end{itemize}
\end{itemize}

\section*{Beispiel 3: ein sehr einfacher Webserver}
\begin{itemize}
  \item eine Schleife, die mittels \code{nc} einen vordefinierten Text liefert
  \item siehe auch: \href{http://de.wikipedia.org/wiki/Netcat}{WP: netcat}
\end{itemize}

\end{document}
